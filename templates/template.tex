%!platex robo-template.tex

%%%%%%%%%%%%%%%%%%%%%%%%%%%%%%%%%%%%%%%%%%%%%%%%%%%%%%%%
% RoboTechテンプレ案。                                 %
% 使いたい機能があったら適宜コメントを外してください。 %
%%%%%%%%%%%%%%%%%%%%%%%%%%%%%%%%%%%%%%%%%%%%%%%%%%%%%%%%

% documentclass
% とりあえず奥村さんのjsarticle
% 他にはjsbook, jreport, thesis等。違いは自分で調べましょう
\documentclass{jsarticle}
% \documentclass{jarticle}
% 余白
\oddsidemargin -3.0mm
\evensidemargin -3.0mm
\textwidth 167mm
\textheight 250mm
\topmargin -15mm
\columnsep 10mm

% 画像基本
\usepackage[dvipdfm]{graphicx,color}

% 図をまわりこんだ文章を書きたいとき
\usepackage{wrapfig}

% 縦にぶちぬく表を作成
\usepackage{multirow}

% 数学関連
\usepackage{amsmath,amssymb,amsthm,bm}

% ハイパーリンク
% \usepackage[dvipdfmx          
% ,colorlinks=true,linkcolor=blue        % リンクの色
% ,bookmarks=true,bookmarksnumbered=true  % しおり
% ,pdftitle={ロボット競技を体験しよう 回路講習会資料},pdfauthor={島津 真人}      % 作者情報
% ]{hyperref}
% しおりを使うときはこれも有効にする
\AtBeginDvi{\special{pdf:tounicode EUC-UCS2}} % EUC
% \AtBeginDvi{\special{pdf:tounicode 90ms-RKSJ-UCS2}} % Shift_JIS

% minipage
\usepackage{boxedminipage}

% 多段組み
%\usepackage{multicol}

% 目次
% \usepackage{makeidx}
% \makeindex

% % 図表の強制位置決め
\usepackage{here}

% 図とcaptionの間隔
\setlength\abovecaptionskip{0pt}
\setlength\belowcaptionskip{0pt}

% 図や表のキャプションを無理やり
% \makeatletter


% ewcommand{\figcaption}[1]{\def\@captype{\figure}\caption{#1}}


% ewcommand{\tblcaption}[1]{\def\@captype{\table}\caption{#1}}
% \makeatother

% 絶対値、ノルム
% \abs{v}, orm{u}などで呼び出せる
\providecommand{\abs}[1]{\lvert#1\rvert}
\providecommand{\orm}[1]{\lVert#1\rVert}

% ページ番号つけたくないとき
%\pagestyle{empty}

% プログラムのソースをlistingsをつかって書く
\usepackage{listings,jlisting}
\usepackage[dvipdfm]{color}
\renewcommand{\lstlistingname}{list}
\lstset{%
language={C++},
backgroundcolor={\color[gray]{.97}},%
tabsize=2,%
basicstyle={\small},%
identifierstyle={\small},%
commentstyle={\small},%
keywordstyle={\small\bfseries},%
ndkeywordstyle={\small},%
stringstyle={\small\ttfamily},
frame={tb},
breaklines=true,
columns=[l]{fullflexible},%
xrightmargin=0zw,%
xleftmargin=0zw,%
numberstyle={\footnotesize},%
numbers=left,%
numbersep=0.2zw,%
stepnumber=1,%
showstringspaces=false,%
lineskip=-0.2ex%
}

%%%%%%%%%%%%%%%%%%%%%%%%%%%%%
% タイトル
\title{振動実験と回転体の危険速度}
\author{曾我 遼}
\date{\today}


%%%%%%%%%%%%%%%%%%%%%%%%%%%%%
\begin{document}

% タイトル書く
\maketitle

% 目次
\tableofcontents
% 図一覧
%\listoffigures
% 表一覧
%\listoftables


% 本文
\newpage
 \section{マイクロガスタービンの運転}
 今回の観察においては、マイクロガスタービンに電灯を3つつなぎ、その音・回転数・排熱の変化をみた。\\
 音は回転数の増減により大きさが決まっていると考えられるため、ここではマイクロガスタービンに対してかかる負荷が増えた時および減った時における音の変化を通して、それぞれの時にどのような運転をしているかを書く。
 電灯0 → 電灯3 すぐに音が大きくなる\\
 電灯3 → 電灯2 音は緩やかに変化し、小さくなる\\
 電灯2 → 電灯3 すぐに音が大きくなる\\
 電灯3 → 電灯0 音は緩やかに変化し、小さくなる\\
 よって、マイクロガスタービンの運動は、負荷が増加する方向に対しての応答性はよいが、負荷が減少する方向に対しての応答性は悪いという事がわかった。

 \section{回転軸系のねじり振動実験}
  \subsection{固有振動数、減衰比の導出式}
  \underline{\textbf{実験値}}\\
  \indent{グラフから読み取った固有角振動数$\omega$から、(3.25)式より慣性モーメントIを求める。}
  \begin{eqnarray}
   I=\frac{k_{r}}{\omega^2}
  \end{eqnarray}
  
  \underline{\textbf{理論値}}\\
  \indent{固有角振動数$\omega$は、式(3.25)により導出できる。}
  \begin{eqnarray}
   \omega=\sqrt{\frac{k_{r}}{I}}
  \end{eqnarray}
  \indent{(3.35)式から、減衰比$\zeta$は、対数減衰比$\delta$との関連で次のように求められる。}
  \begin{eqnarray}
   \zeta=\frac{\delta}{2\pi}=\frac{1}{2\pi}\ln{\frac{a_{n+1}}{a_{n}}}
  \end{eqnarray}
  \subsection{各種値の導出}
  \subsubsection{穴無}
  \begin{table}[H]
   \begin{center}
    \begin{tabular}{|c||c|c|} \hline
        &t(時間) &a(変位) \\ \hline \hline
     n  &0.135   &4.661   \\ \hline
     n+1&0.235   &4.585   \\ \hline
    \end{tabular}
   \end{center}
   \caption{穴無における測定値}
   \label{tbl:ltx-tbl2}
  \end{table}
  \underline{\textbf{実験値}}\\
  \indent{固有周波数$\omega$は、}
  \begin{eqnarray}
   \omega=\frac{2\pi}{0.235-0.135}=62.8
  \end{eqnarray}
  \indent{慣性モーメントIは、次のようになる。}
  \begin{eqnarray}
   I=\frac{1.65\times10}{62.8^2}=4.18\times10^{-3}
  \end{eqnarray}

  \underline{\textbf{理論値}}\\  
  \indent{慣性モーメントIは、軸諸元などの数値により、次のように導出した。}
  \begin{eqnarray}
   I=\frac{1}{2}Ma^2\approx2.01\times10^{-3}
  \end{eqnarray}
  \indent{固有振動数$\omega$は、次のようになる。}
  \begin{eqnarray}
   \omega=\sqrt{\frac{1.65\times10}{2.01\times10^{-3}}}\approx90.7
  \end{eqnarray}
  \indent{減衰比$\zeta$は、}
  \begin{eqnarray}
   \zeta=\frac{\delta}{2\pi}=\frac{1}{2\pi}\ln{\frac{4.585}{4.661}}\approx2.62\times10^{-3}
  \end{eqnarray}

  \subsubsection{穴有}
  \begin{table}[H]
   \begin{center}
    \begin{tabular}{|c||c|c|} \hline
        &t(時間) &a(変位) \\ \hline \hline
     n  &0.025  &3.6191   \\ \hline
     n+1&0.115  &3.5797   \\ \hline
    \end{tabular}
   \end{center}
   \caption{穴有における測定値}
   \label{tbl:ltx-tbl2}
  \end{table}
  \underline{\textbf{実験値}}\\  
  \indent{固有周波数$\omega$は、}
  \begin{eqnarray}
   \omega=\frac{2\pi}{0.115-0.025}=69.78
  \end{eqnarray}
  \indent{慣性モーメントIは、次のようになる。}
  \begin{eqnarray}
   I=\frac{1.65\times10}{62.8^2}=3.39\times10^{-3}
  \end{eqnarray}
  
  \underline{\textbf{理論値}}\\  
  \indent{慣性モーメントIは、軸諸元などの数値により、次のように導出した。}
  \begin{eqnarray}
   I=\frac{1}{2}Ma^2-6I_{穴}=2.01\times10^{-3}-6\times3.52\times10^{-6}\approx1.61\times10^{-3}
  \end{eqnarray}
  \indent{固有振動数$\omega$は、次のようになる。}
  \begin{eqnarray}
   \omega=\sqrt{\frac{1.65\times10}{1.61\times10^{-3}}}\approx101
  \end{eqnarray}
  \indent{減衰比$\zeta$は、}
  \begin{eqnarray}
   \zeta=\frac{\delta}{2\pi}=\frac{1}{2\pi}\ln{\frac{3.5797}{3.6191}}\approx-1.74\times10^{-3}
  \end{eqnarray}
  \subsubsection{穴有+錘}
  \begin{table}[H]
   \begin{center}
    \begin{tabular}{|c||c|c|} \hline
        &t(時間) &a(変位) \\ \hline \hline
     n  &0.055   &-4.525   \\ \hline
     n+1&0.170   &-4.446   \\ \hline
    \end{tabular}
   \end{center}
   \caption{穴有+錘における測定値}
   \label{tbl:ltx-tbl2}
  \end{table}
  \underline{\textbf{実験値}}\\  
  固有周波数$\omega$は、
  \begin{eqnarray}
   \omega=\frac{2\pi}{0.17-0.055}=54.6
  \end{eqnarray}
  \indent{慣性モーメントIは、次のようになる。}
  \begin{eqnarray}
   I=\frac{1.65\times10}{54.6^2}=5.53\times10^{-3}
  \end{eqnarray}
  
  \underline{\textbf{理論値}}\\  
  \indent{慣性モーメントIは、軸諸元などの数値により、次のように導出した。}
  \begin{eqnarray}
   I=\frac{1}{2}Ma^2-6I_{穴}+3I_{錘}=2.01\times10^{-3}-6\times3.52\times10^{-6}+3\times5.63\times10{-4}\approx3.29\times10^{-3}
  \end{eqnarray}
  \indent{固有振動数$\omega$は、次のようになる。}
  \begin{eqnarray}
   \omega=\sqrt{\frac{1.65\times10}{3.29\times10^{-3}}}\approx70.78
  \end{eqnarray}
  \indent{減衰比$\zeta$は、}
  \begin{eqnarray}
   \zeta=\frac{\delta}{2\pi}=\frac{1}{2\pi}\ln{\frac{4.446}{4.525}}\approx2.80\times10^{-3}
  \end{eqnarray}

  \subsubsection{スポンジを挟んだ時(減衰のみ)}
  \begin{table}[H]
   \begin{center}
    \begin{tabular}{|c||c|c|} \hline
        &t(時間) &a(変位) \\ \hline \hline
     n  &0.160   &4.892   \\ \hline
     n+1&0.255   &4.709   \\ \hline
    \end{tabular}
   \end{center}
   \caption{スポンジを挟んだ時における測定値}
   \label{tbl:ltx-tbl2}
  \end{table}
  減衰比$\zeta$は、
  \begin{eqnarray}
   \zeta=\frac{\delta}{2\pi}=\frac{1}{2\pi}\ln{\frac{4.709}{4.892}}\approx-6.07\times10^{-3}
  \end{eqnarray}
  \subsection{理論値との比較}
  \begin{table}[H]
   \begin{center}
    \begin{tabular}{|c|c||c|c|c|} \hline
     % \multicolumn{2}{|c||}{\multirow{2}{*}{捩り振動実験}}
     \multicolumn{2}{|c||}{\multirow{2}{*}{捩り振動実験}}
     & \multicolumn{1}{c|}{慣性モーメント($\times10^{-3}$))} 
     & \multicolumn{1}{c|}{固有角振動数} & \multicolumn{1}{c|}{減衰比($\times10^{-3}$)}\\ \cline{3-5}
     \multicolumn{2}{|c||}{}     
     & \multicolumn{1}{c|}{I} 
     & \multicolumn{1}{c|}{$\omega$} & \multicolumn{1}{c|}{-}\\ \hline \hline
            &穴無    &2.01 &90.7 &- \\ \cline{2-5} 
     理論値 &穴有     &1.61 &101  &- \\ \cline{2-5}
            &穴有+錘 &3.29 &70.78 &- \\ \hline
     \multirow{4}{*}{実験値}
            &穴無     &4.18 &62.8 &-2.62\\ \cline{2-5}
            &穴有     &3.39   &69.78 &\\ \cline{2-5}
            &穴有+錘 &5.53 &54.6 &-2.80\\ \cline{2-5}
            &スポンジ &-       &- &-6.07\\ \hline

    \end{tabular}
   \end{center}
   \caption{実験及び理論値による慣性モーメント、固有角振動数、減衰比の比較}
   \label{tbl:ltx-tbl2}
  \end{table}
  
  理論値と実験値を比較すると、慣性モーメントは実験値が理論値の約2倍となっていることがわかる。\\
  しかし、オーダーは合致しており、また各事象における大小も合っているため、精度は良いと判断できる。
  
 \section{回転軸系の曲げ振動実験}
  \subsection{軸回転数と波形の関連、危険速度前後の回転軸振動の振幅についての考察}
  \subsection{危険速度の理論値と実験値との比較}
  \subsubsection{実験値}
  実験より、各測定値は以下のようになった。

  \begin{table}[H]
   \begin{center}
    \begin{tabular}{|c||c|c|} \hline
             &回転数(rpm) &水平方向変位(mm) \\ \hline \hline
     $\omega_{n}$ &1400   &2.388  \\ \hline
     $\omega_{1}$ &1340   &1.689  \\ \hline
     $\omega_{2}$ &1425   &1.689  \\ \hline

    \end{tabular}
   \end{center}
   \caption{回転系の曲げ振動実験データ}
   \label{tbl:ltx-tbl2}
  \end{table}
  危険速度$\omega_{n}$は、次元をあわせて次の値となる。
  \begin{eqnarray}
   \omega_{n}=\frac{1400\times2\pi}{60}=147[rad/s]
  \end{eqnarray}
  減衰比$\zeta$は、$\omega$を用いて、次のように表せる。
  \begin{eqnarray}
   \zeta=\frac{\omega_{2}-\omega_{1}}{2\omega_{n}}
  \end{eqnarray}
  代入することで、減衰比を得る
  \begin{eqnarray}
   \zeta=\frac{1425-1340}{1400}=6.07\times10^{-2}
  \end{eqnarray}
  \subsubsection{理論値}
  危険速度$\omega_{n}$は、次の式
  \begin{eqnarray}
   \omega_{n}=\sqrt{\frac{k}{m}}=\sqrt{\frac{48EI}{l^{3}m}}
  \end{eqnarray}
  に代入して、
  \begin{eqnarray}
   \omega_{n}=\sqrt{\frac{48\times1.93\times10^{11}\times3.07\times10^{-11}}{0.4^{3}\times0.3352}}\approx115.1
  \end{eqnarray}
  \subsection{比較・考察}
  水平方向の変位と回転数との関係について考える。\\
  実験では、1400rpmのとき変位最大となり、危険速度が1400rpmであると求まったが、この値は理論値と大きく離れている。\\
  これは、理論値を導出する際に用いた式において、両端が自由端であると仮定されていたからであると考えられる。\\
  \\
  危険速度を過ぎると変位は減少していくが、1460rpm周辺において再度増加に転じている。\\
  これは、垂直方向変位に影響を受けたことが原因であると考えられる。
  
\end{document}
